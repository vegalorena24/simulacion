% ----------------------------------------------------------------------------------------------------------------------
%       MPI PRACTISE
% ----------------------------------------------------------------------------------------------------------------------

\documentclass[onecolumn]{article}
\usepackage[utf8]{inputenc}   %con esto, voy a permitir los acentos sin menester del codigo
\usepackage[english]{babel}
%%%%%%%%%%%%%%%%%%%%%%%%%%%%%%%%%%%%%%%%%%%
\usepackage{graphicx,epsfig}
\usepackage{mathtools}
\usepackage{amsfonts,amsmath,amssymb,amsthm} 
\usepackage{relsize}
\usepackage{dcolumn}
\usepackage{array}
\usepackage{graphicx}
\usepackage{ulem}
\usepackage{subfig}
\usepackage{sidecap}
\usepackage{wrapfig}
\usepackage{lipsum}
\usepackage{hyperref}
\usepackage{floatrow}
\usepackage{fancyhdr}
\usepackage{bm} %bold math symbols
\usepackage{hhline}

%%%%%%%%%%%%%%%%%%%%%%%%%%%%%%%%%%%%%%%%%%%
\newfloatcommand{capbtabbox}{table}[][\FBwidth]
\renewcommand{\vec}[1]{\mathbf{#1}} 
\renewcommand{\it}[1]{\textit{#1}}
\renewcommand{\d}{\text{d}}

\begin{document}

% --- Configurar la pagina ---------------------
\pagestyle{fancy}
\lhead{\bf EIA informe}
\rhead{Team}
%\lfoot{ }
\rfoot{Barcelona, March 2017}
% ----------------------------------------------


\title{EIA : Plantilla sencilla y cutre tipo artículo, una columna. Acentos disponibles.}
\author{Author: All of us}
\date{Wednesday, 29th March}

\maketitle

\section{Introduction}

\lipsum

\section{Initialisation of the system}

\lipsum

\section{Forces Elena}

\lipsum

\section{Forces Xabi}

\lipsum

\section{Euler Integrator}

\lipsum

\section{Verlet Integrator}

\lipsum

\section{PBC}

\lipsum

\section{Data}

\lipsum

\section{Visual}

\lipsum




\end{document}
